\section*{Abstract}

Academic funding plays an essential role in research projects. Due to the finite amount of academic funding and an increasing number of emerging research fields, funding agencies could provide more funding to some specific topics, resulting in a topic bias in funding agencies. Combining topic modelling with a supervised machine-learning algorithm, I examined the topic bias from the information of funded projects in the National Institutes of Health (NIH), European Research Council (ERC), National Science Foundation (NSF) and UK Research and Innovation (UKRI). Then, I used a supervised classifier to predict the amount intervals of funding, with topic distributions generated from topic models and project duration as predictors. I collected 145,787 funded project data between 2015 and 2021 from 4 academic funding agencies and extracted topic distributions using latent Dirichlet allocation. Then, I evaluated the topic distributions in each agency and each amount interval of funding per day in project duration. Afterwards, I created Light Gradient Boosting Machine to classify three intervals of funding amounts (i.e low-, medium- and high-amount classes) and made comparisons to models with different predictors in different agencies. The results show that (1) topic modelling is helpful for topic assignment of project abstracts; (2) The topic “Computer Systems and Commercial Application” dominated NSF, ERC and UKRI agencies, whereas the topic “Cancer Treatment and Immunology” dominated NIH agency; (3) With combining the topics ``Computer Systems and Commercial Application" with ``Cancer Treatment and Immunology" (e.g. computational biology or medical engineering), a project could have a higher probability to obtain more funding in all agencies; (4) Topic distributions are found to improve the accuracy of prediction of the funding amount by 5\% -- 20\% once I combined them with project duration as predictors together, even though the model fails to have sufficient prediction power when the model solely considered topic distributions as predictors.

\textbf{Keywords: } Text Mining, Bibliometrics, latent Dirichlet allocation, Multiclassification, Funding agencies, Academic funding, National Institutes of Health, European Research Council, National Science Foundation, UK Research and Innovation, LightGBM, Machine Learning
