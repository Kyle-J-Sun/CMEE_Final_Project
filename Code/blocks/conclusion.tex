\section{Conclusion}

In conclusion, I answer the five questions mentioned in the introduction section. (1) With topic modelling approach, it enables us to obtain the interpretable topics from four academic funding agencies. (2) Except for the topics ``Cancer Treatment and Immunology" in NIH agency and ``Computer System and Commercial Application" in all other agencies, the eight topics have different occurrence probabilities over discrete daily funding amount, meaning that the existence of topic bias in different intervals of daily funding amount and they may provide some information for prediction of the funding amount. (3) The research projects that are relevant to  ``Computer System and Commercial Application" and ``Cancer Treatment and Immunology" possibly like computational epidemiology or medical engineering \textit{etc.} have a higher probability of obtaining more funding in all funding agencies. However, some projects that are related to Environment, Ecosystem and Environment with Education or Social Policy are more likely to be funded with the less daily amount. (4) I found it useful for LightGBM to classify the three classes in both of total and daily funding amount, even though the accuracy is poorer as predicting the discrete daily funding amount in the group of all other agencies. (5) Using topic distributions as predictors is able to boost the accuracy of prediction by approximately 5\% to 20\%.
