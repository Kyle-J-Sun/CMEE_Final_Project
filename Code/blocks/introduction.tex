\section{Introduction}

The advancement of academic research is not only dependent on the advancement of technology, but also on funding support. Academic funding can provide essential support for scientists, especially for medicine and natural science studies \cite{Magua2017}. However, with an increase in the number of researchers, scientists and emerging research fields all over the world \cite{david2014}, the competition of the academic market is increasingly fierce so that funding fail to cover all excellent research projects. This indicates a decrease in opportunities for scientists to obtain enough funding to complete their research projects, many scientists worry about their research careers \cite{Acuna2012}. Therefore, it is important to understand determinants of funding success for research projects and how they will be biased due to different reasons.

In 2011, Allesina \textit{et al.} studied whether nepotism can affect scientific success in Italian Academia using conventional statistical analysis \cite{allesina2011}. They leveraged the Monte Carlo approach to show there are fewer different last names of principal investigators (PIs) compared with the expected situation at random for each discipline in Italian research institutions. This revealed the existence of nepotism in most disciplines, even though the results are underestimated due to the exclusion of cases of nepotism of the mother-child type. Furthermore, they also used logistic regression models to examine how the probability of sharing the last name changes with factors such as geography, institutions, sub-disciplines and latitude, concluding a significant latitudinal influence on nepotism.

However, at that time, in most of the research that studies the factors of scientific success such as what Allesina \textit{et al.} did, they used traditional statistical analysis to uncover the factors that determine the funding success. In recent years, with the development of machine-learning techniques and the computational power of computers, big data are more influential for us in recent decades than in the past \cite{hilbert2016}. It is possible to find the relationship between things and hence predict what will happen in the future through big data in many research fields. For example, Mahajan \textit{et al.} utilised the Na{\"i}ve Bayes model with a large amount of data in the stock market to predict the movement of the stock market and conclude that the model is useful for predicting the future prices of shares \cite{Kannan2016StockMP}.

Therefore, more researchers utilised machine-learning and text mining techniques to extract information and patterns from numerous text data such as scientific papers. In 2014, Emre \textit{et al.} analysed 100,000 publications from Computer Science to examine the centrality in the coauthorship network and use it with a random forest algorithm to predict whether a paper can be highly cited five years after publication \cite{sarigol2014}. The final outcome showed a high accuracy of prediction. Additionally, Michael \textit{et al.} applied several machine-learning classifiers to predict the funding success of projects from a crowdfunding website -- Kickstarter. They used the information of projects such as project duration, the number of Twitter followers and the number of Facebook connections \textit{etc.} as features, obtaining 68\% of the accuracy of prediction \cite{greenberg2013}. Instead of using the basic information of projects exclusively, Tanushree and Eric (2014) extracted phrases in the project description by text-mining techniques from Kickstarter and explored whether they are the factors determining the funding success \cite{mitra2014}.

To explore the impact factors of academic funding success using text-mining approaches, Magua \textit{et al.} (2017) studied the problem of gender bias in the National Institutes of Health (NIH), an academic funding agency. They evaluated the impact on the funding process caused by gender differences in NIH by combining Latent Dirichlet Allocation (LDA) and qualitative analysis \cite{Magua2017}. They first used the regression models to examine the sex differences by assigned scores for male (PIs) and female PIs. The outcome suggested, compared with female PIs, that male PIs' applications have higher priority, approach and significance scores with having the same experience level and research quality. With latent Dirichlet allocation (LDA), they also found that funded male PIs are usually described as ``pioneers" in specific fields with ``highly innovative" and ``significant research", whereas funded female PIs are described as ``expertise" or ``excellent". The outcome revealed a distinct criterion that may lead to different funding success rates of funding applications between male and female PIs.

The difference of research topics of the projects is also another one of the most important impact factors of funding success \cite{Alberts787}. By finding the `hot' topics that can be more funded, researchers can study their own fields by combining these `hot` topics to increase the probability of obtaining more funding. However, using big data approaches to study the relationship of topics and academic funding are not found yet.

Thus, the objective of this project is to explore the relationship between topics and academic funding in the National Institutes of Health (NIH), National Science Foundation (NSF), European Research Council (ERC) and UK Research and Innovation (UKRI), which are four well-known academic funding agencies in the world. At the end of the project, I try to answer the following questions:

\begin{itemize}
    \item Is text-mining techniques useful for finding `hot' topics in different agencies?
    \item Whether different topics could affect the funding amount in four different academic funding agencies?
    \item What are some topics that could improve the probability of obtaining more funding in four agencies?
    \item Are academic funding can be predicted using Machine Learning Approach?
    \item Can topics distributions improve the accuracy of prediction of academic funding amount?
\end{itemize}
